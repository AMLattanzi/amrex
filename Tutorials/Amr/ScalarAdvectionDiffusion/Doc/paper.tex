\documentclass{article}

\bibliographystyle{plain}

\usepackage{epsfig}
\usepackage{amssymb}
\usepackage{amsmath}

\newcommand{\ibold}{{\bf i}}
\newcommand{\jbold}{{\bf j}}
\newcommand{\ebold}{{\bf e}}
\newcommand{\vbold}{{\bf v}}
\newcommand{\xbold}{{\bf x}}
\newcommand{\ubold}{{\bf u}}
\newcommand{\nbold}{{\bf n}}
\newcommand{\Fbold}{{\bf F}}
\newcommand{\ubar}{{\bar {\bf u}}}
\newcommand{\vb}{{\bf u}}
\newcommand{\npo}{{n + 1}}
\newcommand{\dt}{{\Delta t}}
\newcommand{\dx}{{h}}
\newcommand{\nph}{{n + \frac{1}{2}}}
\newcommand{\half}{\frac{1}{2}}
\newcommand{\iph}{{\ibold + \half \ebold^d}}
\newcommand{\imh}{{\ibold - \half \ebold^d}}


\title{Scalar Advection-Diffusion Example in AMReX}

\author{Dan Graves}
\begin{document}

\begin{abstract}
  The advection-diffusion example is intended to explore how one does
  a higher order algorithm in AMReX.    The example consists of an
  $\tt{AmrLevel}$-derived class with several variations of
  advection-diffusion algorithms implemented and tested.  These
  include a second-order Godunov algorithm, as well second, third, and
  fourth order Runge-Kutta.   Diffusion terms are updated explicitly,
  making this example only appropriate for sufficiently small
  diffusion coefficients.
\end{abstract}

\section{Introduction}

Given a velocity field $\vbold$, a scalar field  $\phi$ and a
diffusion coefficient $\nu$, the standard way to write the 
scalar advection-diffusion equation  is given by 
$$
\frac{\partial \phi}{\partial t} + \nabla \cdot (\vbold \phi) = \nabla
(\nu \nabla \phi).
\cot 
$$
For this application, we are assuming that 
\begin{itemize}
\item The diffusion coefficient is constant.  This would be fairly
  trivial to change.
\item For any grid spacing $\dx$ we intend to us, the diffusion
  coefficient $\nu$ is sufficiently small that the time
  step $\dt$ is controlled by the advective CFL condition
$$
\dt < \frac{\dx}{v_m}.
$$
where $v_m= \max|\vbold|$;
Roughly speaking, this means that 
$$
\nu < v_m \dx.
$$
\end{itemize}
Given that we will updating the diffusion explicitly, we can
reformulate the equation as 
$$
\frac{\partial \phi}{\partial t} =  -\nabla \cdot \Fbold, 
$$
where $\Fbold = \vbold \phi - \nu \nabla phi$.
The algorithms here are implemented in finite volume form:
$$
D(F)_\ibold = \frac{1}{\dx} \sum^D_{d = 1} (F_{\iph} - F_\imh)
$$
and are
therefore strongly conservative.

\section{Algorithm Choices}

\subsection{Second Order Godunov} 

In second-order Godunov, we extrapolate in space and time to get a
flux at the half time step.   I will not go into great detail here
because this is fairly old hat to my audience.
$$
\phi_{i+1/2} = \phi_i + \half\dx \partial{\phi}{\partial x} + \half\dt \partial{\phi}{\partial t}
$$
We use the equation of motion to substitute spatial derivatives for
the time derivative and we get.   We use  limited van Leer slopes to
approximate the spatial derivatives.  We extrapolate to every face
from both sides and pick the upwind state.   For the diffusive flux,
we simply take a centered difference of $\phi^n$ at the face.  So, for
diffusion, in this algorithm, we are first order in time.   Since we
are only working with small diffusion coefficients, it does not seem
to matter to the convergence numbers.   For coarse-fine interpolation,
this algorithm just uses FillPatch.

\subsection{RK2}

The second-order RK scheme uses two evaluations of $\nabla \cdot F$ at
fixed times to advance the solution
$$
\begin{array}{l}
\phi^\nph = \phi^n -  \frac{\dt}{2} D(F)^n\\
\phi^{n+1} = \phi^n - \dt D(F)^\nph
\end{array}
$$
We evaluate the divergence operator using spatial extrapolation and
upwinding.
$$
\begin{array}{l}
\phi_{i+1/2, L} = \phi_i + \frac{\dx}{2} \Delta^{vl,x} \phi_i, \\
\phi_{i+1/2, R} = \phi_{i+1} - \frac{\dx}{2} \Delta^{vl,x} \phi_{i+1}
\end{array}
$$
where $\Delta^{vl,x}(\phi)$ is the van Leer limited slope of $\phi$ in
the $x$ direction.   To compute the flux at the face, we pick the
upwind state then multiply by the normal velocity at the face.   
 For coarse-fine interpolation,  this algorithm just uses FillPatch.

\subsection{RK3 (TVD)}

For our third-order RK3 TVD  integrator, we  use what Gottlieb and Shu
\cite{gottlieb_shu} cite as the optimal third order TVD RK3
integrator.
$$
\begin{array}{l}
\phi^1 = \phi^n - \dt D(F(\phi^n)) \\
\phi^2 = \frac{3}{4} \phi^n + \frac{1}{4} (\phi^1 - \dt D(F(\phi^1)))\\
\phi^{n+1} = \frac{1}{3} \phi^n +  \frac{2}{3}(\phi^2 - dt D(F(\phi^2)))
\end{array}
$$
Our spatia discretization is discribed by McCorquodale and Colella
\cite{mccorquodale_collela} and is fourth order in space.   As of the
writing of this document,  we have implemented the simpler version
presented there.
\begin{itemize}
  \item Get pointwise $\phi$ because I will need this to get higher
    order gradient at faces.  
$$
\phi = <\phi> -\frac{h^2}{24}(L(<\phi>)),
$$
   where $L$ is the standard second order Laplacian.
\item Get the average of the scalar over the faces .
$$
<\phi>_{i+1/2} =  \frac{7}{12}(<\phi>_i    + <\phi>_{i+1})
                -\frac{1}{12}(<\phi>_{i-1} + <\phi>_{i+2})
$$

\item Get the point values of the scalar on faces from the face
  average:
$$
\phi_{i+1/2} = <\phi>_{i+1/2} - \frac{1}{24} L^T(<phi>)_{i+1/2},
$$
where $L^T$ is the Laplacian that does not include terms normal to the
face (what McCorquodale refers to as the ``transverse Laplacian'').

\item Compute pointwise fluxes $F$ (including diffusion fluxes).
$$
F_{i+1/2} = u_{i+1/2} \phi_{i+1/2}  - nu G^x(phi)_{i+1/2}
$$
where the normal gradient $G^x$ is computed using fourth order finite
differences  of the pointwise cell-centered scalar $\phi$.
$$
G^x(\phi)_{i+1/2} = \frac{\partial \phi}{\partial x} + O(h^4) 
= (1/h)(\phi_{i+1}-\phi_{i}) 
- (1/(4h))((\phi_{i+2} + \phi_i - 2\phi_{i+1}) - (\phi_{i+1} + \phi_{i-1} - 2\phi_{i}))
$$

\item Transform the pointwise fluxes $F$ into face-averaged fluxes $<F>$
$$
<F> = F - (1/24) L^T(F).
$$    
\end{itemize}
My next task is to implement the more complicated algorithm described
in \cite{mccorquodale_collela}, which includes limiting and Riemann
problems.   For coarse-fine interpolation, we use FillPatchUtil
quartic interpolation for spatial interplation.   For time
interpolation,     We form  a polynomial in time (following Fok and
Rosales \cite{fok_rosales}).
$$
\phi(t^{l-1} + \xi \dt^{l-1}) = \phi^0 + \xi k_1 + \frac{\xi^2}{2}(k_2-k_1)
$$
where $k_1 = -D(F(\phi^n)), k_2 = -D(F(\phi^1))$.

\subsection{RK4}

This option has precisely the same discretization as McCorquodale.
The only exception being the spacial coarse-fine interpolation is done
via FillPatchUtil using quartic interpolation.

\bibliography{references}

\end{document}

