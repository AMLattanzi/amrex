\section{Running in Parallel}
We will now demonstrate how to run the example in {\tt BoxLib/Tutorials/WaveEquation\_C/}
in parallel.  On your local machine, if you have MPI installed, you can simply build
the executable as describe before, but with {\tt USE\_MPI=TRUE} in the {\tt GNUmakefile}.
Alternatively, you can override the settings in {\tt GNUmakefile} at the command line
using, e.g., ``{\tt make USE\_MPI=TRUE}''.  
An executable named {\tt main2d.Linux.g++.gfortran.MPI.ex} will be built.
Then you can run the program in parallel using, e.g.,
\begin{lstlisting}[backgroundcolor=\color{light-red}].
mpiexec -n 4 main2d.Linux.g++.gfortran.MPI.ex inputs_2d
\end{lstlisting}

To run in parallel on the hopper machine at NERSC, first copy the \BoxLib\ source code
into your home directory on hopper and enter the {\tt WaveEquation\_C/} directory.
The default programming environment uses the {\tt PGI} compilers, so we will switch to the
{\tt gnu} programming environment to make {\tt g++} and {\tt gfortran} available
using the command:
\begin{lstlisting}[backgroundcolor=\color{light-red}].
module swap PrgEnv-pgi PrgEnv-gnu
\end{lstlisting}
Next, in {\tt GNUmakefile}, set {\tt USE\_MPI=TRUE}, and then type ``{\tt make}''
(or alternatively, type ``{\tt make USE\_MPI=TRUE}'').
An executable named {\tt main2d.Linux.g++.gfortran.MPI.ex} will be built.
You cannot submit jobs in your home directory, so change to a scratch space
(``{\tt cd \$SCRATCH}" will typically do), and copy the executable and
{\tt inputs\_2d} into this directory.  Then you need to create a job script,
e.g., "{\tt hopper.run}", that has contents:
\lstinputlisting[backgroundcolor=\color{light-red}]{./AdvancedTopics/hopper.run}
To run, simply type ``{\tt qsub hopper.run}''.  You can monitor the status of your job
using ``{\tt qstat -u <username>}'' and view your position in the queue 
using ``{\tt showq}''.

\section{Boundary Conditions}

\section{Nodal Data}

\section{Checkpoints and Restarting}

\section{Linear Solvers}

\section{Adaptive Mesh Refinement}

\section{OpenMP and Threads}
