There are several visualization tools that can be used for \amrex\
plotfiles.  The popular \visit\ package supports the \amrex\ file 
format natively (using the {\sf BoxLib2d} and {\sf BoxLib3d} file types),
as does the \yt\ python package.  The standard tool used within the
\amrex-community is \amrvis, a package developed and supported 
by CCSE that is designed specifically for highly efficient visualization
of block-structured hierarchical AMR data.

\section{\amrvis}

\begin{enumerate}

\item Download and build \amrvis:
\begin{verbatim}
git clone https://ccse.lbl.gov/pub/Downloads/Amrvis.git
\end{verbatim}

Then {\tt cd} into {\tt Amrvis/}, edit the {\tt GNUmakefile} by
setting {\tt COMP} to the compiler suite you have.

Type {\tt make DIM=2} or {\tt make DIM=3} to build, 
resulting in an executable that looks like {\tt amrvis2d...ex}.

If you want to build amrvis with {\tt DIM=3}, you must first
download and build {\tt volpack}:
\begin{verbatim}
git clone https://ccse.lbl.gov/pub/Downloads/volpack.git
\end{verbatim}

Then {\tt cd} into {\tt volpack/} and type {\tt make}.

Note: \amrvis\ requires the OSF/Motif libraries and headers. If you don't have these 
you will need to install the development version of motif through your package manager. 
{\tt lesstif} gives some functionality and will allow you to build the amrvis executable, 
but \amrvis\ may exhibit subtle anomalies.

On most Linux distributions, the motif library is provided by the
{\tt openmotif} package, and its header files (like {\tt Xm.h}) are provided
by {\tt openmotif-devel}. If those packages are not installed, then use the
OS-specific package management tool to install them. 

You may then want to create an alias to {\tt amrvis2d}, for example
\begin{verbatim}
alias amrvis2d /tmp/Amrvis/amrvis2d...ex
\end{verbatim}

\item Generally the plotfiles have the form {\tt *pltXXXXX} 
  (the prefix can be changed), where {\tt XXXXX} is a number 
  corresponding to the timestep the file
  was output.  {\tt amrvis2d <filename>} or {\tt amrvis3d <filename>}
  to see a single plotfile, 
  or for 2D data sets, {\tt amrvis2d -a *plt*}, which will animate the 
  sequence of plotfiles.

  Try playing
  around with this---you can change which variable you are
  looking at, select a region and click ``Dataset'' (under View)
  in order to look at the actual numbers, etc. You can also export the
  pictures in several different formats under "File/Export".

  We have created a number of routines to convert \amrex\ plotfile data
  other formats (such as MATLAB), but in order to properly interpret 
  the hierarchical AMR data, each tends to have its own idiosyncrasies.
  If you would like to display the data in another format, please contact
  Marc Day ({\tt MSDay@lbl.gov}) and we will point you to whatever we have
  that can help.

\section{\visit}

\amrex\ data can also be visualized by {\tt VisIt}
(\url{https://wci.llnl.gov/simulation/computer-codes/visit/}), an open
source visualization and analysis software.

\section{\yt}

{\tt yt}, an open source python package available at
\url{http://yt-project.org/}, can be used for analyzing and
visualizing mesh and particle data generated by \amrex\ codes.  Some
of the \amrex\ developers are also yt project members.

\end{enumerate}
