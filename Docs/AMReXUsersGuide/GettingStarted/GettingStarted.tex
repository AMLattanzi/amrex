In this chapter, we will walk you through two simple examples.  It is
assumed here that your machine has GNU Make, Python, GCC (including
gfortran), and MPI, although \amrex\ can be built with CMake and other
compilers. 

\section{Downloading the Code}

The source code of \amrex\ is available at
\url{https://github.com/AMReX-Codes/amrex}.  The GitHub repo is our
central repo for development.  The {\tt development} branch
includes the latest state of the code, and it is merged into the {\tt
  master} branch on a monthly basis.  The {\tt master} branch is
considered the release branch.  The releases are tagged with version
number {\tt YY.MM} (e.g., {\tt 17.04}).  The {\tt MM} part of the
version is incremented every month, and the {\tt YY} part every year.
Bug fix releases are tagged with {\tt YY.MM.patch} (e.g., {\tt
  17.04.1}).

\section{Example: Hello World}

The source code of this example is at {\tt
  amrex/Tutorials/Basic/HelloWorld\_C/} and is also shown below. 

\begin{lstlisting}[language=cpp]
 #include <AMReX.H>
 #include <AMReX_Print.H>

 int main(int argc, char* argv[])
 {
     amrex::Initialize(argc,argv);
     amrex::Print() << "Hello world from AMReX version " 
                    << amrex::Version() << "\n";
     amrex::Finalize();
 }
\end{lstlisting}

The main body of this short example contains three statements.
Usually the first and last statements for the {\tt main} function of
every program should be calling {\tt amrex::}\idxamrex{Initialize} and
\idxamrex{Finalize}, respectively.  The second statement calls {\tt
  amrex::}\idxamrex{Print} to print out a string that includes the
\amrex\ version returned by the {\tt amrex::}\idxamrex{Version}
function.  The example code includes two \amrex\ header files.  Note
that the name of all \amrex\ header files starts with {\tt AMReX\_}
(or just {\tt AMReX} in the case of {\tt AMReX.H}).  All \amrex\
\cpp\ functions are in the {\tt amrex} namespace.  

\subsection{Building the Code}

You build the code in the {\tt amrex/Tutorials/Basic/HelloWorld\_C/}
directory.  Typing {\tt make} will start the compilation process and
result in an executable named {\tt main3d.gnu.DEBUG.ex}.  The name
shows that the GNU compiler with debug options set by \amrex\ is used.
It also shows that the executable is built for 3D.  Although this
simple example code is dimension independent, the dimension matters
for all non-trivial examples.  The build process can be adjusted by
modifying the {\tt amrex/Tutorials/Basic/HelloWorld\_C/GNUmakefile} file.
More details on how to build \amrex\ can be found in
Chapter~\ref{Chap:BuildingAMReX}.

\subsection{Running the Code}

The example code can be run as follows,
\begin{verbatim}
  ./main3d.gnu.DEBUG.ex
\end{verbatim}
The result may look like,
\begin{verbatim}
  Hello world from AMReX version 17.05-30-g5775aed933c4-dirty
\end{verbatim}
The version string means the current commit {\tt 5775aed933c4} (note
that the first letter {\tt g} in {\tt g577..} is not part of the hash)
is based on {\tt 17.05} with 30 additional commits and the \amrex\
work tree is dirty (i.e. there are uncommitted changes).

In the {\tt GNUmakefile} there are compilation options for {\tt DEBUG}
mode (less optimized code with more error checking), dimensionality,
compiler type, and flags to enable MPI and/or OpenMP parallelism.
If there are multiple instances of a parameter, the last instance
takes precedence.

\subsection{Parallelization}

Now let's build with MPI by typing {\tt make USE\_MPI=TRUE} (alternatively
you can set {\tt USE\_MPI=TRUE} in the {\tt GNUmakefile}).  This
should make an executable named {\tt main3d.gnu.DEBUG.MPI.ex}.  Note
{\tt MPI} in the file name.  You can then run,
\begin{verbatim}
  mpiexec -n 4 ./main3d.gnu.DEBUG.MPI.ex
\end{verbatim}
The result may look like,
\begin{verbatim}
  MPI initialized with 4 MPI processes
  Hello world from AMReX version 17.05-30-g5775aed933c4-dirty
\end{verbatim}

If the compilation fails, you are referred to
Chapter~\ref{Chap:BuildingAMReX} on how to configure the build
system.

If you want to build with OpenMP, type {\tt make USE\_OMP=TRUE}.
This should make an executable named {\tt main3d.gnu.DEBUG.OMP.ex}.  Note
{\tt OMP} in the file name.  Make sure the {\tt OMP\_NUM\_THREADS}
environment variable is set on your system.  You can then run,
\begin{verbatim}
  ./main3d.gnu.DEBUG.OMP.ex
\end{verbatim}
The result may look like,
\begin{verbatim}
  OMP initialized with 4 OMP threads
  Hello world from AMReX version 17.06-287-g51875485fe51-dirty
\end{verbatim}
Note that you can build with both {\tt USE\_MPI=TRUE} and {\tt USE\_OMP=TRUE}.
You can then run,
\begin{verbatim}
  mpiexec -n 2 ./main3d.gnu.DEBUG.MPI.OMP.ex
\end{verbatim}
The result may look like,
\begin{verbatim}
  MPI initialized with 2 MPI processes
  OMP initialized with 4 OMP threads
  Hello world from AMReX version 17.06-287-g51875485fe51-dirty
\end{verbatim}

\section{Example: Heat Equation Solver}\label{sec:heat equation}

We now look at a more complicated example at
{\tt amrex/Tutorials/Basic/HeatEquation\_EX1\_C} and show how simulation
results can be visualized.  This example solves the heat equation,
\begin{equation}
\frac{\partial\phi}{\partial t} = \nabla^2\phi
\end{equation}
using forward Euler temporal integration on a periodic domain.  
We could use a 5-point (in 2D) or 7-point (in 3D) stencil, but for demonstration
purposes we spatially discretize the PDE by first constructing fluxes on cell faces, e.g.,
\begin{equation}
F_{i+\myhalf,j} = \frac{\phi_{i+1,j}-\phi_{i,j}}{\Delta x},
\end{equation}
and then taking the divergence to update the cells,
\begin{equation}
\phi_{i,j}^{n+1} = \phi_{i,j}^n 
+ \frac{\Delta t}{\Delta x}\left(F_{i+\myhalf,j}-F_{i-\myhalf,j}\right)
+ \frac{\Delta t}{\Delta y}\left(F_{i,j+\myhalf}-F_{i,j-\myhalf}\right)
\end{equation}
Don't worry about the implementation details of the code.
You will be able to understand the code in this example after
Chapter~\ref{Chap:Basics}. 

\subsection{Building and Running the Code}

To build a 2D executable, type {\tt make DIM=2}.  This will generate
an executable named {\tt main2d.gnu.ex}.  To run it, type,
\begin{verbatim}
  ./main2d.gnu.DEBUG.ex inputs_2d
\end{verbatim}
Note that the command takes a file {\tt inputs\_2d}.  When the run
finishes, you will have a number of plotfiles, {\tt plt00000}, {\tt
  plt01000}, etc.  The calculation solves the heat equation in 2D on a
$256 \times 256$ cells domain.  It runs $10,000$ steps and makes a
plotfile every $1,000$ steps.  These are runtime parameters that can
be adjusted in {\tt inputs\_2d}.

\section{Visualization}
There are several visualization tools that can be used for \amrex\
plotfiles.  The standard tool used within the
\amrex-community is \amrvis, a package developed and supported 
by CCSE that is designed specifically for highly efficient visualization
of block-structured hierarchical AMR data.
Plotfiles can also be viewed using the \visit, \paraview, and \yt\ packages.
Particle data can be viewed using \paraview.
Refer to Chapter \ref{Chap:Visualization} for how to use each of these tools.
