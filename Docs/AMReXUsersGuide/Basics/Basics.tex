In this chapter, we present the basics of \amrex.  The implementation
source codes are in {\tt amrex/Src/Base/}.  After reading this
chapter, one should be able to understand many of the example codes in
{\tt amrex/Tutorials}, and one should be able to develop single-level
parallel codes using \amrex.  Note that \amrex\ classes and functions
are in namespace {\tt amrex}.  For clarity, we usually drop {\tt
  amrex::} in the example codes here.

\section{Dimensionality}
\label{sec:basics:dim}

As we have mentioned in Chapter~\ref{Chap:BuildingAMReX}, the
dimensionality of \amrex\ must be set at compile time.  A macro, {\tt
  BL\_SPACEDIM}, is defined to be the number of spatial dimensions.
The coordinate directions are zero based. \MarginPar{Not in Fortran}

\section{Print}
\label{sec:basics:print}

\amrex\ provides classes for printing messages to standard output or
any \cpp\ {\tt ostream}.  The main reason one should use them instead
of {\tt std::cout} is that messages from multiple processes or
threads do not get mixed up.  Below are some examples.
\begin{lstlisting}[language=cpp]
    Print() <<  "x = " << x << "\n"; // Print on I/O processor

    AllPrint() << "Every process prints\n";  // Print on every process

    std::ofstream ofs("my.txt", std::ofstream::out);
    Print(ofs) << "Print to a file" << std::endl;
    ofs.close();
\end{lstlisting}

\section{Box, IntVect and IndexType}
\label{sec:basics:box}

In \amrex, the computational domain (on each AMR level) is decomposed
into a union of rectangular domains.  {\tt Box} is the data
structure for representing such rectangular domains in indexing space.
{\tt Box} is a dimension dependent class.  It has lower and upper
corners (represented by {\tt IntVect} and an index type
(represented by {\tt IndexType}).

\subsection{IntVect}

{\idxamrex{IntVect}} is a dimension dependent class representing an
integer vector in {\idxamrex{BL\_SPACEDIM}}-dimensional space.  An
{\tt IntVect} can be constructed as follows,
\begin{lstlisting}[language=cpp]
    IntVect iv(D_DECL(19, 0, 5));
\end{lstlisting}
Here {\tt D\_DECL} is a macro that expands {\tt D\_DECL(19,0,5)} to
either {\tt 19} or {\tt 19,0} or {\tt 19,0,5} depending on the number
of dimensions.  The class has a static function {\tt
  IntVect::TheZeroVector()} returning the zero vector, {\tt
  IntVect::TheUnitVector()} returning the unit vector, and {\tt
  TheDimensionVector(int dir)} returning a reference to a constant
{\tt IntVect} that is zero except in the {\tt dir}-direction.  {\tt
  IntVect} has a number of relational operators, {\tt ==}, {\tt !=},
{\tt <}, {\tt <=}, {\tt >}, and {\tt >=} that can be used for
lexicographical comparison.  It also has various arithmetic operators.
For example,
\begin{lstlisting}[language=cpp]
    IntVect iv(D_DECL(19, 0, 5));
\end{lstlisting}


\subsection{IndexType}

\subsection{Box}

% a bit on coarsening

\section{Array}

\section{BaseFab and FArrayBox}

\section{BoxArray}

\section{DistributionMapping}

\section{ParallelDescriptor}

\section{FabArray and MultiFab}

\section{MFIter and Tiling}

