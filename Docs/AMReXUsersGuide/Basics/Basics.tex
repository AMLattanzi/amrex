In this chapter, we discuss the basics of \amrex.  The implementation
source codes are in {\tt amrex/Src/Base/}.  After reading this
chapter, one should be able to understand many of the example codes in
{\tt amrex/Tutorials}, and one should be able to develop parallel
single-level codes using \amrex.

\section{Box, IntVect and IndexType}
\label{sec:basics:box}

In \amrex, the computational domain (on each AMR level) is decomposed
into a union of rectangular domains.  {\tt Box} is the data
structure for representing such rectangular domains in indexing space.
{\tt Box} is a dimension dependent class.  It has lower and upper
corners (represented by {\tt IntVect} and an index type
(represented by {\tt IndexType}).

\subsection{IntVect}

{\idxamrex{IntVect}} is a dimension dependent class representing an
integer vector in {\idxamrex{BL\_SPACEDIM}}-dimensional space, where {\tt
  BL\_SPACEDIM} is the number of spatial dimension.

\subsection{IndexType}

\subsection{Box}

\section{BaseFab and FArrayBox}

\section{BoxArray}

\section{DistributionMapping}

\section{FabArray and MultiFab}

\section{MFIter}

