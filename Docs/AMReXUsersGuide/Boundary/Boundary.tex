This chapter describes how to implement domain boundary conditions in \amrex.
A ghost cell that is outside of the valid region can be thought of as either
``interior'' (for periodic and coarse-fine ghost cells), or ``physical''.
Physical boundary conditions can include inflow, outflow, slip/no-slip walls,
but are ultimately linked to mathematical Dirichlet or Neumann conditions.

The basic idea behind physical boundary conditions is as follows:
\begin{itemize}

\item Create a {\tt BCRec} object, which is essentially a multidimensional integer array of
{\tt 2*DIM} components.  Each component defines a boundary condition type for
the lo/hi side of the domain, for each direction.
See {\tt Src/Base/AMReX\_BC\_TYPES.H} for common physical and mathematical types.
If there is more than one variable, we can create an array of {\tt BCRec} objects,
and pass in a pointer to the 0-index component since the arrays for all the 
components are contiguous in memory.
Here we need to provide boundary types to each component of the {\tt
MultiFab}.  Below is an example of setting up {\tt Vector<BCRec>}
before the call to ghost cell routines.
\begin{lstlisting}[language=cpp]
  // Set up BC; see Src/Base/AMReX_BC_TYPES.H for supported types
  Vector<BCRec> bc(phi.nComp());
  for (int n = 0; n < phi.nComp(); ++n)
  {
      for (int idim = 0; idim < AMREX_SPACEDIM; ++idim)
      {
          if (Geometry::isPeriodic(idim))
          {
              bc[n].setLo(idim, BCType::int_dir); // interior
              bc[n].setHi(idim, BCType::int_dir);
          }
          else
          {
              bc[n].setLo(idim, BCType::foextrap); // first-order extrapolation
              bc[n].setHi(idim, BCType::foextrap);
          }
      }
  }
\end{lstlisting}
{\tt amrex::BCType} has the following types,
\begin{quote}
\begin{description}
\item [int\_dir] Interior, including periodic boundary
\item [ext\_dir] ``External Dirichlet''.  It is the user's responsibility to write a routine
to fill ghost cells (more details below).
\item [foextrap] ``First Order Extrapolation'' 
First order extrapolation from last cell in interior.
\item [reflect\_even] Reflection from interior cells with sign
  unchanged, $q(-i) = q(i)$.
\item [reflect\_odd] Reflection from interior cells with sign
  unchanged, $q(-i) = -q(i)$.
\end{description}
\end{quote}

\item We have interfaces to a fortran routine that fills ghost cells at domain
boundaries based on the boundary condition type defined in the {\tt BCRec} object.
It is the user's responsibility to have a consisent definition of what the ghost cells
represent.  A common option used in \amrex\ codes is to fill the domain ghost cells
with the value that lies on the boundary (as opposed to another common option where
the value in the ghost cell represents an extrapolated value based on the boundary
condition type).  Then in our stencil based ``work'' codes, we also pass in the
{\tt BCRec} object and use modified stencils near the domain boundary that know the value
in the first ghost cell represents the value on the boundary.

\end{itemize}

Depending on the level of complexity of your code, there are various options
for filling domain boundary ghost cells.

For single-level codes built from {\tt Src/Base} (excluding the 
{\tt Src/AmrCore} and {\tt Src/Amr} source code directories), you will have 
single-level {\tt MultiFab}s filled with data in the valid region where you need 
to fill the ghost cells on each grid.  There are essentially three ways to fill the ghost 
cells. (refer to Tutorials/Basic/HeatEquation\_EX2\_C for an example).

\begin{lstlisting}[language=cpp]
MultiFab mf;
Geometry geom;
Vector<BCRec> bc;

// ...

// fills interior and periodic domain boundary ghost cells
mf.FillBoundary(geom.periodicity());

// fills interior (but not periodic domain boundary) ghost cells
mf.FillBoundary();

// fills physical domain boundary ghost cells
FillDomainBoundary(mf, geom, bc);
\end{lstlisting}

{\tt FillDomainBoundary()} is a function is in {\tt Src/Base/AMReX\_BCUtil.cpp},
and is essentially an interface to fortran {\tt subroutine amrex\_fab\_filcc()}
in {\tt Src/Base/AMReX\_filcc\_mod.F90}, which ultimately calls fortran 
{\tt subroutine filcc()} in {\tt Src/Base/AMReX\_FILCC\_XD.F}.  To create more
custom boundary conditions, create a local modified copy of
{\tt Src/Base/AMReX\_FILCC\_XD.F} and put it your local source code.

For multi-level codes using the {\tt Src\_AmrCore} source code, the
functions described above still work, however additional classes need to
be set up since the {\tt FillPatch} routines call them.
In fact it is possible to avoid using the single-level calls directly if
you fill all your grids and ghost cells using the {\tt FillPatch} routines.
Refer to {\tt Tutorials/Amr/Advection\_AmrCore/} for an example.
The class {\tt PhysBCFunct} in {\tt Src/Base/AMReX\_PhysBCFunct.cpp}
is derived from {\tt PhysBCFunctBase} and contains a {\tt BCRec}, {\tt Geometry},
and a pointer to a {\tt BndryFunctBase} function.

The function {\tt FillBoundary} fills physical ghost cells (so it has a different
functionality from the single-level case described above, where {\tt FillDomainBoundary}
fills the physical ghost cells).
