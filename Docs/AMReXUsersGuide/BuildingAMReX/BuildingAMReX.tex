
In this chapter, we discuss \amrex's build systems.  There are three
ways to use \amrex.  The approach used by \amrex\ developers uses GNU
Make.  There is no installation step in this approach.  Application
codes adopt \amrex's build system and compile \amrex\ while compiling
their own codes.  This will be discussed in more details in
Section~\ref{build:make}.  The second approach is to build \amrex\
into a library and install it (Section~\ref{build:lib}).  Then an
application code uses its own build system and links \amrex\ as an
external library.  \amrex\ can also be built with CMake
(Section~\ref{build:cmake}).

\section{Building with GNU Make}
\label{build:make}

In this build approach, you write your own make files defining a
number of variables and rules.  Then you invoke {\tt make} to start
the building process.  An example of this can be found in {\tt
  amrex/Tutorials/Basic/HelloWorld\_C}.  We will 

xxxxx the variables we can set

xxxxx make.package

rules, help, make.machines, make.local xxx 

\section{Building {\tt libamrex}}
\label{build:lib}

\section{Building with CMake}
\label{build:cmake}

\MarginPar{todo}

