\amrex\ supports local ODE integration using the \cvode\ solver,\footnote{\url{https://computation.llnl.gov/projects/sundials/cvode}} which is part of the \sundials\ framework.\footnote{\url{https://computation.llnl.gov/projects/sundials}}
\cvode\ contains solvers for stiff and non-stiff ODEs, and as such is well suited for solving e.g., the complex chemistry networks in combustion simulations, or the nuclear reaction networks in astrophysical simulations.

Most of \cvode\ is written in C, but many functions also come with two distinct Fortran interfaces.
One interface is \fcvode, which is bundled with the stable release of \cvode.
Its usage is described in the \cvode\ documentation.\footnote{\url{https://computation.llnl.gov/sites/default/files/public/cv_guide.pdf}}
However, the use of \fcvode\ is discouraged in \amrex\ due to its incompatibility with being used inside OpenMP parallel regions (which is the primary use case in \amrex\ applications).

The alternative, and recommended, Fortran interface to \cvode\ uses the \texttt{iso\_c\_binding} feature of the Fortran 2003 standard to implement a direct interface to the C functions in \cvode.
When compiling \cvode, one need not build the \cvode\ library with the \fcvode\ interface enabled at all.
Rather, the Fortran 2003 interface to \cvode\ is provided within \amrex\ itself.
The \cvode\ tutorials provided in \amrex\ use this new interface.

\section{Compiling \cvode}

To use \cvode\ in an \amrex\ application, follow these steps:

\begin{enumerate}

  \item Obtain the \cvode\ source code, which is hosted here: \url{https://computation.llnl.gov/projects/sundials/sundials-software}.

  One can download either the complete \sundials\ package, or just the \cvode\ components.

  \item Unpack the \cvode/\sundials\ tarball, and create a new ``build'' directory (it can be anywhere).

  \item Navigate to the new, empty build directory, and type

  \begin{verbatim}
  cmake \
    -DCMAKE_INSTALL_PREFIX:PATH=/path/to/install/dir \
    /path/to/cvode/or/sundials/top/level/source/dir
  \end{verbatim}

  The \texttt{CMAKE\_INSTALL\_DIR} option tells CMake where to install the libraries.
  Note that CMake will attempt to deduce the compilers automatically, but respects certain environment variables if they are defined, such as \texttt{CC} (for the C compiler), \texttt{CXX} (for the C++ compiler), and \texttt{FC} (for the Fortran compiler).
  So one may modify the above CMake invocation to be something like the following:

  \begin{verbatim}
  CC=/path/to/gcc \
  CXX=/path/to/g++ \
  FC=/path/to/gfortran \
    cmake \
    -DCMAKE_INSTALL_PREFIX:PATH=/path/to/install/dir \
    /path/to/cvode/or/sundials/top/level/source/dir
  \end{verbatim}

  One can supply additional flags to CMake or to the compiler to customize the compilation process. 
  Flags of interest may include \texttt{CMAKE\_C\_FLAGS}, which add the specified flags to the compile statement, e.g.,
  \texttt{-DCMAKE\_C\_FLAGS="-h list=a"} will append the \texttt{-h list=a} flag to the \texttt{cc} statement when compiling the source code.
  Here one may wish to add something like \texttt{"-O2 -g"} to provide an optimized library that still contains debugging symbols; 
  if one neglects debugging symbols in the \cvode\ library, and if a code that uses \cvode\ encounters a segmentation fault in the solve, 
  then the backtrace has no information about where in the solver the error occurred.
  Also, if one wishes to compile only the solver library itself and not the examples that come with the source 
  (compiling the examples is enabled by default), one can add \texttt{"-DEXAMPLES\_ENABLE=OFF"}.
  Users should be aware that the \cvode\ examples are linked dynamically, so when compiling the solver library on Cray system using the Cray compiler wrappers \texttt{cc}, \texttt{CC}, and \texttt{ftn}, one should explicitly disable compiling the examples via the \texttt{"-DEXAMPLES\_ENABLE=OFF"} flag.

  \item In the \texttt{GNUmakefile} for the \amrex\ application which uses the Fortran 2003 interface to \cvode, add \texttt{USE\_CVODE = TRUE}, which will compile the Fortran 2003 interfaces and link the \cvode\ libraries.
  Note that one must define the \texttt{CVODE\_LIB\_DIR} environment variable to point to the location where the libraries are installed.
\end{enumerate}


\section{The \cvode\ Tutorials}

\amrex\ provides two \cvode\ tutorials in the \texttt{Tutorials/CVODE} directory, called \texttt{EX1} and \texttt{EX2}.
\texttt{EX1} consists of a single ODE that is integrated with \cvode\ within each cell of a 3-D grid.
It demonstrates how to initialize the \cvode\ solver, how to call the ODE right-hand-side (RHS), and, more importantly, how to \emph{re-}initialize the solver between cells, which avoids allocating and freeing solver memory between each cell (see the call to \texttt{FCVReInit()} in the \texttt{integrate\_ode.f90} file in the \texttt{EX1} directory.)

The \texttt{EX2} example demonstrates the slightly more complicated case of integrating a system of coupled ODEs within each cell.
Similarly to \texttt{EX1}, it provides an RHS and some solver initialization.
However, it also demonstrates the performance effect of providing an analytic Jacobian matrix for the system of ODEs, 
rather than requiring the \cvode\ solver to compute the Jacobian matrix numerically using a finite-difference approach.
The tutorial integrates the same system of ODEs on the same 3-D grid, but in one sweep it instructs \cvode\ 
to use the analytic function that computes the Jacobian matrix, and in the other case, it does not,
which requires \cvode\ to compute it manually.
One observes a significant performance gain by providing the analytic Jacobian function.
