The core of \amrex\ is written in \cpp.  For Fortran users who want to
write all of their programs in Fortran, \amrex\ provides Fortran
interfaces around most of functionalities except for particles
(Chapter~\ref{Chap:Particles}) and the {\tt AmrLevel} class
(Chapter~\ref{Chap:AmrLevel}).  We should not confuse the Fortran
interface in this chapter with the Fortran kernel functions called
inside {\tt MFIter} loops in \cpp codes
(Section~\ref{sec:basics:fortran}).  For the latter, Fortran is used
in some sense as a domain-specific language with native
multi-dimensional arrays, whereas here Fortran is used to drive the
whole application code.  In order to better understand \amrex, Fortran
interface users should read the rest of the User's Guide except for
Chapters~\ref{Chap:Particles} \& \ref{Chap:AmrLevel}. 

\section{Getting Started}

We have discussed \amrex's build systems in
Chapter~\ref{Chap:BuildingAMReX}.  To build with GNU Make, we need to
include the Fortran interface source tree into the make system.  The
source codes for the Fortran interface are in {\tt
amrex/Src/F\_Interfaces} and there are several sub-directories.  The
{\tt Base} directory includes sources for the basic functionality, the
{\tt AmrCore} directory wraps around {\tt AmrCore} class
(Chapter~\ref{Chap:AmrCore}), and the {\tt Octree} adds support for
octree type of AMR grids.  Each directory has a {\tt Make.package}
file that can be included in make files (see {\tt
Tutorials/Basic/HelloWorld\_F} and {\tt Tutorials/Amr/Advection\_F}
for examples).  The {\tt libamrex} approach includes the Fortran
interface by default.  The CMake approach does not support the Fortran
interface yet.



\section{The Basics}

\section{Amr Core Infrastructure}

\section{Octree}
