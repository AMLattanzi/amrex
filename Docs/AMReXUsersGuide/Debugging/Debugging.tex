Debugging is an art and everyone has their own preferred method.  In
this chapter, we will present a few techniques that might be useful
for \amrex\ users.

\section{Debug Build, Floating-Point Exceptions and Backtrace}
\label{sec:debug:fpe}

There are a lot of assertions in \amrex.  These assertions are turned
off for optimized build.  When debugging, we find it is helpful to
build with debug options.  For example, for building with GNU Make
(Chapter~\ref{Chap:BuildingAMReX}), one can type {\tt make DEBUG=TRUE}
to turn on debug options.  In addition to assertions, this will also
add various compiler options such as floating pointer exception
trapping, fortran array bound checking, etc.  For floating pointer
exception trapping, one should use {\tt ParmParse} parameter {\tt
amrex.fpe\_trap\_invalid=1}.  (Note that one can append this option to
the command line without modifying the inputs file.)  This parameter
can be used to initialize all {\tt MultiFab} data to signaling NaNs.
If an assertion fails or a floating point exception occurs, backtrace
files named {\tt Backtrace.*} will be generated, one for each parallel
process.  We can then examine the contents of these backtrace files
and hopefully they can tell us where in the code the problem occurs.

% send a signal

% Compile with Backtrace = True

% section on tools like valgrind, git bisect

% section on vismf diffmf fcompare


