We now proceed through a series of tutorial applications of increasing complexity
to help understand how to write your own application.  Any of these examples
can be used as starting points for your own application.  Each of the examples
will be presented in fortran and C++.  The examples are contained in the \BoxLib\
release found at {\tt https://ccse.lbl.gov/pub/BoxLib/}.

\section{Diffusion Equation - Fortran}
Our first example, contained in {\tt BoxLib/Tutorials/WaveEquation\_F}, advances the equations
\begin{eqnarray}
\frac{\partial\phi_1}{\partial t} &=& \phi_2, \\
\frac{\partial\phi_2}{\partial t} &=& \nabla^2\phi_1,
\end{eqnarray}
on a domain from $[-1,-1,-1]$ to $[1,1,1]$ (in 3D; analogous in 2D) with initial conditions
\begin{eqnarray}
\phi_1(t=0) &=& 0 \\
\phi_2(t=0) &=& e^{-100r^2},
\end{eqnarray}
where $r$ is the distance from the cell-center to the center of the domain.
We use uniform grid spacing in each direction, i.e., $\Delta x = \Delta y = \Delta z$,
and for this example, we used a fixed time step with $\Delta t = 0.1\Delta x$.
To advance these equations, we use a Runge-Kutta temporal discretization and a 
fourth-order spatial discretization of the Laplacian.  This example does not use AMR, 
and uses periodic boundary conditions on all sides.

In the problem directory, you will see the following files:
\begin{itemize}
\item {\tt GNUmakefile}

This contains compiler settings and directories required by the make system to build the code.

  \begin{itemize}

    \item {\tt BOXLIB\_HOME}

    Change this to point to the \BoxLib\ home directory.  Alternatively, you can define {\tt BOXLIB\_HOME}
    as an environment variables on your system.

    \item {\tt MKVERBOSE}

    Verbosity of compile-time output.

    \item {\tt NDEBUG}
      
    ``Not Debug'' (we know, confusing).  If 't', modifies compiler flags to build a more optimized version
    of the code.

    \item {\tt MPI}

    Indicate whether you want your executable to be MPI-compatible.  In general there is no reason to turn this
    off since you can still run the resulting executable serially.

    \item {\tt OMP}

    Turns on OpenMP compiler flags.  Note that you still must write OpenMP directives into your code.

    \item {\tt COMP}

    The compiler.  Supported options include Intel, PathScale, and PGI.  gfortran seems to be bug-free
    on all systems we've run on.  Intel after version 9 seems flaky.  PathScale (available at OLCF and NERSC)
    seems to work as long as you don't turn the optimization flags too high.  PGI seems to work fine, but
    is slower than the others.

  \end{itemize}

\item {\tt GPackage.mak}

List of local files needed to be included in the build.  The {\tt GNUmakefile} points to this.

\item {\tt main.f90}, {\tt init\_data.f90}, {\tt advance.f90}, {\tt write\_plotfile.f90}

Source code that is not within the {\tt BoxLib/Src/} tree.  Note: if a file that exists in the
{\tt BoxLib/Src/} tree also exists in the local directory, the local copy takes precedence.
{\tt main.f90} is the driver program that calls subroutines from the other fortran90 files.

\item {\tt inputs\_2d}, {\tt inputs\_3d}

Inputs files to customize the simulation parameters.

\end{itemize}

To build the code, simply type ``make''.  An exectubale will appear that has some indication (but not complete)
about what setting you used in the {\tt GNUmakefile}.  To run the code serially, simply type, for example,
\begin{verbatim}
./main.Linux.gfortran.mpi.exe inputs_2d
\end{verbatim}
To run in parallel, if you have MPI on your machine, instead type, for example,
\begin{verbatim}
mpiexec -n 4 ./main.Linux.gfortran.mpi.exe inputs_2d
\end{verbatim}
In the parallel case,you should see the following on your screen at the beginning
\begin{verbatim}
MPI initialized with            4  MPI processes
MPI initialized with            1  threads
\end{verbatim}
The program will complete and there will be a series of plotfiles, e.g., {\tt plt00000}, in the run directory.
You can open these using {\tt VisIt} (available at {\tt https://wci.llnl.gov/codes/visit/}) by opening
the {\tt Header} file within the plotfile directory.  (For the {\tt VisIt} novice, after you open the {\tt Header}
file, select ``Add'' $\rightarrow$ ``Pseudocolor'' $\rightarrow$ ``Variable 2'' and then click ``Draw''.)
