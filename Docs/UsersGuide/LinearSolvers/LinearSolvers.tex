\section{\BoxLib\ Multigrid Philosophy}

Here we describe some of the general ideas behind the \BoxLib\
multigrid (MG).

We solve MG on an AMR hierarchy, which means in places we will encounter
C-F interfaces.  The \BoxLib\ MG will modify the stencil for the Laplacian
at C-F interfaces to ensure the correct solution to the Poisson equation
across these interfaces.  There is no correction step needed in this
case (this differs from Ricker, ApJs (2008).

The MG solver always works with jumps of $2\times$ between levels.  In 
some codes, we can have jumps of $4\times$ between
levels.  \BoxLib\ uses a {\tt mini\_cycle} in these cases, effectively
inserting a multigrid level between the two AMR levels and doing a mini
V-cycle.

The MG solvers are located in {\tt BoxLib/Src/LinearSolvers/F\_MG/}.
There are two MG solvers, for cell-centered and nodal data.
Generally, the routines specific to the cell-centered solver will have
{\tt cc} in their name, and those specific to the nodal solver will have
{\tt nd} in their name.

Support for $\Delta x \ne \Delta y \ne \Delta z$




\section{Data Structures}

\subsection{\mgtower}

The \mgtower\ is a special Fortran derived type that carries all the
information required for a \BoxLib\ multigrid solve.

The following parameters are specified when building the \mgtower:
\begin{itemize}
\item {\tt smoother}: the type of smoother to use.  Choices are listed
  in {\tt mg\_tower.f90}.  Common options are {\tt
    MG\_SMOOTHER\_GS\_RB} for red-black Gauss-Seidel and {\tt
    MG\_SMOOTHER\_JACOBI} for Jacobi.

\item {\tt nu1}: The number of smoothings at each level on the way down
  the V-cycle.

\item {\tt nu2}: The number of smoothings at each level on the way up
  the V-cycle.

\item {\tt nub}: The number of smoothing before and after the bottom solver.

\item {\tt gamma}: \MarginPar{what does this mean?}

\item {\tt cycle\_type}: The type of multigrid to do, V-cycles ({\tt
  MG\_VCycle}), W-cycles ({\tt MG\_WCycle}), full multigrid ({\tt
  MG\_FCycle}).

\item {\tt omega}:  \MarginPar{description?}

\item {\tt bottom\_solver}: the type of bottom solver to use.  See the next
 section.

\item {\tt bottom\_max\_iter}: the maximum number of iterations for the 
  bottom solver

\item {\tt bottom\_solver\_eps}: the tolerance used by the bottom
  solver.  

\item {\tt max\_iter}: the maximum number of multigrid cycles.

\item {\tt max\_bottom\_nlevel}: additional coarsening if you use
  {\tt bottom\_solver} type 4 (see below)

\item {\tt min\_width}: minimum size of grid at coarsest multigrid level

\item {\tt rel\_solver\_eps}: the relative tolerance of the solver.

\item {\tt abs\_solver\_eps}: the absolute tolerance of the solver.

\item {\tt verbose}: the verbosity of the multigrid solver.  Higher
  numbers give more verbosity.

\item {\tt cg\_verbose}: the verbosity of the bottom solver. Higher
  numbers give more verbosity.

\end{itemize}

In addition to these parameters, the \mgtower\ carries a number of
\multifab s that carry the solution and stencil for the multigrid
solve.

\begin{itemize}

\item {\tt ss}: The stencil itself---for each zone, this gives the 
  coefficients of the terms in the Laplacian, with the convection that
  the `0' term is located in the current cell and the other terms are
  the $\pm 1$ off the current cell in each direction.

\item {\tt cc}: scratch space (?)

\item {\tt ff}: The source (righthand side) of the elliptic equation
  we are solving.

\item {\tt dd}: The residual/defect, $f - L\phi$

\item {\tt uu}: The solution variable ($\phi$ when on the finest level)

\item {\tt mm}: For cell-centered, {\tt mm} takes a direction and
   tells you whether we are at a Dirichlet or Neumann boundary, or if 
   we are skewed in that direction.

   For nodal, {\tt mm} simply tells us whether a point is Dirichlet
   or Neumann.  There is no skew in nodal.

\end{itemize}


\subsection{\bndryreg}

There are two types of \bndryreg\ objects, each of which is a set of 
{\tt dim} multifabs.  The first, which is built with 
the {\tt bndry\_reg\_build} call, is defined on all cells immediately outside
of each grid.  Each multifab within the \bndryreg\ contains all the fabs on both
the low and high faces for a given direction.  In the context of 
multigrid, we fill the \bndryreg\ with coarse data before interpolating the data
to the correct locations to be used in the multigrid stencil.  The second type
of \bndryreg\ object is defined on the coarse cells immediately outside each
fine grid, and is defined with the  {\tt bndry\_reg\_rr\_build} call, is defined on all cells immediately outside
.  For this latter type, the option is available to include only cells 
not covered by a different fine grid, but this is left as an option because it
requires additional calculations of box intersections.

In multigrid, we use the latter type of \bndryreg\ to calculate the 
residual at coarse cells adjacent to fine grids that is used as the right hand side
in the relaxation step at the coarse level on the way down the V-cycle (or other).  
The first type of \bndryreg is used to hold boundary conditions for the fine grids 
that are interpolated from the coarse grid solution; this is filled on the way back up the V-cycle.

To compute the residual in a coarse cell adjacent to a fine grid, we first
compute the pure coarse residual, then subtract the contribution from the coarse
face underlying the coarse-fine interface, and add the contribution of the fine
faces at the coarse-fine interface.  The \bndryreg\ holds the cell-centered
contribution from the difference of these edge fluxes and is added to
the coarse residual multifab.

\section{Stencils}

There are several different stencil types that we can use for 
the discretization.  For cell-centered, these are:
\begin{itemize}
\item {\tt CC\_CROSS\_STENCIL}: this is the standard cross-stencil---5 points 
in 2-d and 7 points in 3-d.  For cell-centered MG, this is the default, and
is usually the best choice.

\item {\tt HO\_CROSS\_STENCIL}: this is a cross-stencil that uses 9 points
in 2-d and 11-points in 3-d.  For instance, it will use $(i,j)$;
$(i\pm1,j)$; $(i\pm2,j)$; $(i,j\pm1)$; $(i,j\pm2)$.  This is
higher-order accurate than {\tt CC\_CROSS\_STENCIL}.

\item {\tt HO\_DENSE\_STENCIL}: this is a dense-stencil---it uses all the
points (including corners) in $(i\pm1,j\pm1)$, resulting in a 9-point stencil
in 2-d and 27-point stencil in 3-d.
\end{itemize}

For the nodal solver, the choices are:
\begin{itemize}
\item {\tt ND\_CROSS\_STENCIL}: this is the standard cross-stencil.

\item {\tt ND\_DENSE\_STENCIL}: this is a dense stencil, using
  all the points in $(i\pm1,j\pm1)$.  The 
  derivation of this stencil is based on finite-element ideas, defining
  basis functions on the nodes.  This is developed in 2-d in 
  Almgren, Bell, and Szymczack (1996).

\item {\tt ND\_VATER\_STENCIL}: this is an alternate dense stencil derived
  using a similar finite-element idea as above, but a different control
  volume.
\end{itemize}
For the cell-centered solve, the coefficients for the stencil are computed
once, at the beginning of the solve.  For the nodal solver, the coefficients
are hard-coded into the smoothers.

\section{Smoothers}

The following smoothers are available (but not necessarily for both the
cell-centered and nodal solvers):
\begin{itemize}
\item {\tt MG\_SMOOTHER\_GS\_RB}: a red-black Gauss-Seidel smoother

\item {\tt MG\_SMOOTHER\_JACOBI}: a Jacobi smoother (not implemented for 
the dense nodal stencil)

\item {\tt MG\_SMOOTHER\_MINION\_CROSS}

\item {\tt MG\_SMOOTHER\_MINION\_FULL}

\item {\tt MG\_SMOOTHER\_EFF\_RB}
\end{itemize}



\section{Cycling}

The default cycling is a V-cycle, but W-cycles and full multigrid are
supported as well.


\section{Bottom Solvers}

The multigrid cycling coarsens the grids as part of the solve.  When 
the coarsest grid is reached, the individual boxes that comprise that
level are coarsened as much as then can, down to $2^3$ zones.  Depending
on the distribution of sizes of the grids, it may not be possible for
everything to reach this minimum size.  At this point, the bottom
solver is invoked.  Most of these will solve the linear system
on this collection of grids directly.  There is one special bottom
solver that will define a new box encompassing all of the coarsened
grids and then put the data on fewer boxes and processors and further
coarsen the problem, again until we get as close to $2^3$ as possible.
At that point, one of the other bottom solvers will be called upon
to solve the problem.

There are several bottom solvers available in \BoxLib.  
The allowed values are:
\begin{itemize}

\item {\tt mg\_bottom\_solver} / {\tt hg\_bottom\_solver = 0}: smoothing only.

\item {\tt mg\_bottom\_solver} / {\tt hg\_bottom\_solver = 1}: biconjugate
  gradient stabilized---this is the default.

\item {\tt mg\_bottom\_solver} / {\tt hg\_bottom\_solver = 2}: conjugate
  gradient method

\item {\tt mg\_bottom\_solver} / {\tt hg\_bottom\_solver = 4}: a special 
  bottom solver that extends the range of the multigrid coarsening
  by aggregrating coarse grids on the original mesh together and
  further coarsening.

\end{itemize}

You should use the special bottom solver (4) whenever possible, even
if it means changing your gridding strategy (as discussed below) to 
make it more efficient.

\MarginPar{any simple discussion on why we might choose one of the
  other bottom solvers?}

\subsection{Special Bottom Solver}

The special solver takes the data from the coarsest level of the
original multigrid V-cycle and copies it onto a new grid structure with
the same number of total cells in each direction, but with a fewer
number of larger grids.  A new V-cycle begins from this point, so we
are essentially coarsening this ``new'' problem.  Now, the coarsest
level of the multigrid V-cycle in the ``new'' problem has fewer cells
and fewer grids as compared to the original coarsest level.

To enable this solver, set {\tt hg\_bottom\_solver = 4} (for the nodal
projections) and/or {\tt mg\_bottom\_solver = 4} (for the
cell-centered projections) in your inputs file.

To understand how this bottom solver works, the first thing you need
to know is what the grid structure of the coarsest level of your
multigrid V-cycle looks like.  Next, figure out the size of the box you
would need if you wanted it to fit all the data on the coarsest level.
Finally, figure out what the largest integer $n$ is so that you can evenly
divide the length of this box by $2^n$ in every coordinate direction.
If $n < 2$, the program will abort since the grid structure is not
suitable for this bottom solver.

The code will set up a ``new'' problem, using the data at the
coarsest level of the original problem as the initial data.  The grid
structure for this new problem has the same number of cells as the
coarsest level of the original problem, but the data is copied onto a
grid structure where each grid has $2^n$ cells on each side.  The new
V-cycle continues down to the new coarsest level, in which each grid
has 2 cells on each side.  If you wish to impose a limit on the
maximum value that $n$ can have, you can do so by setting {\tt
max\_mg\_bottom\_nlevs} equal to that value.

Some grid examples help make this clear:
\begin{itemize}
\item {\bf Example 1:} A 3D problem with $384^3$ cells divided into $32^3$
grids, i.e., there is a $12\times 12\times 12$ block of $32^3$ grids.
The coarsest level of the multigrid V-cycle contains $12\times
12\times 12$ grids that have $2^3$ cells, so the entire problem domain
has $24^3$ cells.  We see that $n=3$, and create a new problem domain
with a $3\times 3\times 3$ block of $8^3$ grids.  The coarsest level
of the multigrid V-cycle for the ``new'' problem will be a $3\times
3\times 3$ block of $2^3$ grids.

\item {\bf Example 2:} A 2D problem with $96\times 384$ cells divided into
$48^2$ grids, i.e., there is a $2\times 8$ block of $48^2$ grids.  The
coarsest level of the multigrid V-cycle contains $2\times 8$ grids
that have $3^2$ cells, so the entire problem domain has $6\times 24$
cells.  We see that $n=0$, so the program aborts since this grid
structure is not appropriate for the fancy bottom solver.
\end{itemize}




\section{Flowchart}



\subsection{Cell-Centered MG}

The flowchart below shows the structure of a cell-centered multigrid
solve using pure V-cycles.


\begin{itemize}

\item {\tt stencil\_fill\_cc\_all\_mglevels} / {\tt stencil\_fill\_cc}:
  Compute all of the stencil coefficients
  for the Laplacian operator at all cells.  At the C-F interfaces, the
  stencil coefficients are modified to know this.

\item {\tt ml\_cc}: The main driver for the cell-centered multigrid.
   Among other things, this computes the norm that will be used
   for convergence testing.

\item {\tt mg\_tower\_v\_cycle} (recursive):
\begin{itemize}

  \item {\em recursively descend V-cycle}

  \begin{itemize} 
  \item {\tt cc\_mg\_tower\_smoother}: Smooth the 
    problem at the current MG level using the desired smoother.

  \item {\tt compute\_defect}: Construct $f - L\phi$.

  \item {\tt mg\_tower\_restriction}:  Restrict
    the defect to the coarser level by conservative averaging.
  \end{itemize}

  \item {\tt mg\_tower\_bottom\_solve}:  Solve the coarsened problem
    using the chosen bottom solver.

  \item {\em ascend V-cycle}

  \begin{itemize}
  \item {\tt mg\_tower\_prolongation}: Take the solution at level $n-1$ and use it to 
     correct the solution at level $n$ by representing the data on the finer grid.  This uses
     linear reconstruction for jumps by $2\times$ and piecewise-constant otherwise.

  \item {\tt cc\_mg\_tower\_smoother}:
  \end{itemize}

\end{itemize}
\item {\tt compute\_defect}: This is called multiple times, checking for
   convergence at each level.


\end{itemize}



\subsection{Nodal MG}

The flowchart below shows the structure of a cell-centered multigrid
solve using pure V-cycles.


\begin{itemize}

\item {\tt stencil\_fill\_cc\_all\_mglevels} / {\tt stencil\_fill\_cc}:
  For the nodal solver, this applies the weights to the 
  coefficients. \MarginPar{nodal does things differently than cc---need more here}

\item {\tt ml\_nd}: The main driver for the nodal multigrid.

\item {\tt mg\_tower\_v\_cycle} (recursive):
\begin{itemize}

  \item {\em recursively descend V-cycle}
  \begin{itemize} 
  \item {\tt nodal\_smoother\_?d}: Smooth the 
    problem at the current MG level using the desired smoother.

  \item {\tt compute\_defect}: Construct $f - L\phi$.

  \item {\tt mg\_tower\_restriction}:  Restrict
    the defect to the coarser level by simply taking the fine value that
    lies at the same place as the coarse data.
  \end{itemize}

  \item {\tt mg\_tower\_bottom\_solve}:  Solve the coarsened problem
    using the chosen bottom solver.

  \item {\em ascend V-cycle}

  \begin{itemize}
  \item {\tt mg\_tower\_prolongation}: For nodal data, the fine grid
     will have some points at exactly the same place as the coarse data---these are
     simply copied to the fine grid.  The remain data is interpolated.

  \item {\tt modal\_smoother\_?d}:
  \end{itemize}

\end{itemize}
\item {\tt compute\_defect}: This is called multiple times, checking for
   convergence at each level.


\end{itemize}
