We now enhance our heat equation example from Section \ref{sec:HeatEq_1}.
Below is an outline of how we will proceed.  Each of these sections contains an 
accompanying tutorial code that builds upon the previous example.
\begin{itemize}

\item In Section \ref{Sec:Boundary Conditions_C} we develop the
  capability to handle other (non-periodic) boundary condition types.

\end{itemize}

\section{Boundary Conditions}\label{Sec:Boundary Conditions_C}
In order to understand how to implement boundary conditions, we shall
first describe the general principles behind working with boundary
conditions.  The {\tt BoxLib/Tutorials/HeatEquation\_EX2\_C/} tutorial
continues our heat equation example, but now with some non-periodic
boundary condition support.  The boundary condition code in 
{\tt FILCC\_xD.F} and {\tt bc\_fill\_nd.f90} can be used as a springboard
for developing your own customzied boundary conditions.

\subsection{General Principles}

The basic idea behind physical (i.e., non-periodic) 
boundary conditions (such as walls, inflow, and outflow)
is to fill ghost cells with appropriate values, and then possibly supply
alternate discretizations near the boundaries.  For periodic domain boundary conditions,
as well as interior grid - interior grid internal boundaries, the function 
{\tt FillBoundary} fills ghost cells from neighboring valid data:
\begin{verbatim}
// Fill the ghost cells of each grid from the other grids
// includes periodic domain boundaries
old_phi.FillBoundary(geom.periodicity());
\end{verbatim}

The convention for many \BoxLib codes (including this tutorial) is that
the first ghost cell represents the value on the boundary, not the extrapolated
value to the ghost cell-center.  In this case, we incorporate special discretizations
near the boundary that have altered stencils.  
For example, a gradient on a face is typically the difference between two neighboring
cell-centers, divided by {\tt dx}.  On a domain boundary face, we may instead specify that
the gradient is the difference between the neighboring cell-centers, but divided
by {\tt 0.5*dx} since the ghost value represents the value on the boundary face.
This is purely convention, and it is perfectly possible to write code that uses 
the same stencil everywhere (even near the boundary) with ghost cells filled appropriately.







% Here we will discuss some features of the C++ in greater detail. In 
% particular we will examine \BoxLib's passively advected particles.

% \section{Particles}\label{Sec:Particles}
% [Note from Ethan: This section is currently being expanded. Expect information to be 
% incomplete and occasionally inaccurate]

% \subsection{ParticleBase and Particle}
% {\tt ParticleBase} provides the data {\tt struct} upon which all 
% particles are built. Each particle has integer values {\tt m\_id}, {\tt 
% m\_cpu}, {\tt m\_lev}, and {\tt m\_grid} that store the particles id, 
% host cpu, grid level and grid id respectively. It also has an {\tt 
% IntVect} that stores its cell position and a $d$ length real vector 
% storing its spatial position.

% {\tt ParticleBase} also provides basic methods and methods for interfacing between 
% the particle and the grids. These include methods that check the 
% particle position, transfer the particle between grids, and 
% interpolate data appropriately.

% {\tt Particle<N>} is a template class that adds a (real) metadata array of 
% user-specified size to the {\tt ParticleBase}. For instance, this 
% metadata might be 
% \begin{enumerate}
%    \item particle mass
%    \item particle x velocity
%    \item particle y velocity
%    \item particle z velocity
% \end{enumerate}
% \subsection{ParticleContainer}
% {\tt ParticleContainer} is a template class that stores a number of 
% {\tt Particle<N>}s. These particles are stored on a level by level and 
% grid by grid basis to make accessing and performing operations on them 
% more memory efficient. The class also provided methods that move and 
% redistribute the stored particles.
% \section{State Data}
